% {{{ Preamble ----------------------------------------------------------------
\documentclass{beamer}

% encodings, fonts etc.
\usepackage[utf8x]{inputenc}
\usepackage[T1]{fontenc}

% math packages
\usepackage{amsmath, amssymb, amsthm}
\usepackage{mathtools}

% beamer configuration
\usetheme{Copenhagen}
%\usetheme{metropolis}
\beamertemplatenavigationsymbolsempty
%\setbeamertemplate{theorems}[numbered]
\setbeamersize{description width=2.0em}

%\usepackage{pgfpages}
%\setbeameroption{show notes}
%\setbeameroption{show notes on second screen=right}

% algorithms
\usepackage{algorithm}
\usepackage[noend]{algpseudocode}

% misc. packages
\usepackage{float}
\usepackage{varwidth}
\usepackage{listings}
\lstset{
  %breaklines=true,
  keepspaces=true,
  %frame=ltrb,
  %framesep=1pt,
  %commentstyle=\color{grey},
  basicstyle=\ttfamily\tiny,
  numbers=left,
  title=\lstname,
  columns=fullflexible,
  inputencoding=utf8,
  extendedchars=true,
}

% graphics and tikz
\usepackage{pgf}
\usepackage{tikz}
\usetikzlibrary{arrows,automata,trees,positioning}

% mathematics
\newtheorem{proposition}{Proposition}

\renewcommand{\tt}{\texttt}

% title page
\title{Implement a MIPS processor using SME}
\author[Carl-Johannes Johnsen]{
  \mbox{Carl-Johannes Johnsen}}
\institute{Department of Computer Science\\
           University of Copenhagen}
%\date{December 22, 2016}
% }}} -------------------------------------------------------------------------

\begin{document}

% {{{ Title page --------------------------------------------------------------
\frame{\titlepage}
% }}} -------------------------------------------------------------------------

% {{{ Table of contents -------------------------------------------------------
%\begin{frame}
%  \frametitle{Outline}
%  \tableofcontents
%\end{frame}
% }}} -------------------------------------------------------------------------


\begin{frame}{Introduction}
    The goal is to implement a MIPS processor using SME

    Will follow the same procedure as the old ARK course on DIKU
\end{frame}

\begin{frame}{Single cycle}
    The first step is to construct a single cycle MIPS processor, i.e. in one
    clock cycle, exactly one instruction is executed.
    \begin{center}
        \includegraphics[width=0.8\linewidth]{processor.jpg}
    \end{center}
\end{frame}

\begin{frame}{Register file}
    \lstinputlisting[firstline=267,lastline=296,firstnumber=285]{../sme/src/Examples/SingleCycleMIPS/ID.cs}
\end{frame}

\begin{frame}{Write buffer}
    \lstinputlisting[firstline=26,lastline=49,firstnumber=26]{../sme/src/Examples/SingleCycleMIPS/WB.cs}
\end{frame}

\begin{frame}{ALU}
    \lstinputlisting[firstline=202,lastline=220,firstnumber=202]{../sme/src/Examples/SingleCycleMIPS/EX.cs}
    \lstinputlisting[firstline=261,lastline=263,firstnumber=261]{../sme/src/Examples/SingleCycleMIPS/EX.cs}
\end{frame}

\begin{frame}{Control Unit}
    \lstinputlisting[firstline=238,lastline=262,firstnumber=239]{../sme/src/Examples/SingleCycleMIPS/ID.cs}
\end{frame}

\begin{frame}{ALU control}
    \lstinputlisting[firstline=161,lastline=184,firstnumber=161]{../sme/src/Examples/SingleCycleMIPS/EX.cs}
\end{frame}

\begin{frame}{Splitter and Sign extend}
    \lstinputlisting[firstline=163,lastline=179,firstnumber=163]{../sme/src/Examples/SingleCycleMIPS/ID.cs}
    \lstinputlisting[firstline=206,lastline=209,firstnumber=206]{../sme/src/Examples/SingleCycleMIPS/ID.cs}
\end{frame}

\begin{frame}{Instruction memory}
    \lstinputlisting[firstline=98,lastline=111,firstnumber=98]{../sme/src/Examples/SingleCycleMIPS/IF.cs}
\end{frame}

\begin{frame}[fragile]{Test program and output}
    \lstinputlisting[firstline=78,lastline=96,firstnumber=78]{../sme/src/Examples/SingleCycleMIPS/IF.cs}
    After the program has run, the register file has the following contents:
    \begin{lstlisting}[numbers=none]
0, 5, 2, 7, 12, 24, 19, 5, 7, 1, 0, 9, 15, 0,  0,  0,
0, 0, 0, 0, 0,  0,  0,  0, 0, 0, 0, 0, 0,  0,  0,  0
    \end{lstlisting}
\end{frame}

\begin{frame}{Future work}
    \begin{itemize}
        \item Construct the Memory unit
        \item Add support for Jump instructions
        \item Add more instructions
        \item Pipelining
    \end{itemize}
\end{frame}

% exit section
\AtBeginSection{}
\section*{}

% {{{ Bibliography ------------------------------------------------------------
%\begin{frame}{Bibliography}
%  \tiny
%  \bibliographystyle{plain}
%  \bibliography{pl}
%\end{frame}
% }}} -------------------------------------------------------------------------

\end{document}
