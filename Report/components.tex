\subsection{Register file}
The register file is the component that holds values for the processor. It is
the first step in a memory hierarchy, and thus the fastest memory available.
There are 32 registers in a 32-bit MIPS processor. The registers are devided
into groups based on their usage. This does not matter from a hardware
perspective, except for register 0, which is immutable and always 0.

A register file has 5 inputs: Read address A, Read address B, Write enabled,
Write address and Write data. It also has two outputs: Output A and Output B. A
register file writes before reading, as an instruction assumes that the data
potentially produced by an previous instruction is available for the current
instruction.

\bf{Testing} - I start by testing if some of the initial values of the register
file is correctly set to 0. Then I test if I can write to a register, and
whether or not I can read the same value from the same address in the register.
Then, I try to write to all of the registers, except for 0, and check whether
or not the output that I get, corresponds with the output that I wrote.
