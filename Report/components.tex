\subsection{Register file}
The register file is the component that holds values for the processor. It is
the first step in a memory hierarchy, and thus the fastest memory available.
There are 32 registers in a 32-bit MIPS processor. The registers are devided
into groups based on their usage. This does not matter from a hardware
perspective, except for register 0, which is immutable and always 0.

A register file has 5 inputs: Read address A, Read address B, Write enabled,
Write address and Write data. It also has two outputs: Output A and Output B.
We need to be careful of the order in which we read and write from the register
file. We need to make sure that when an instruction reads from the register
file, it always gets the latest data, i.e. if an instruction reads from the
same register as a previous instruction writes to, it should get the written
value. This is easy to fix in the single cycle processor, as we just need to
write before reading.

\begin{figure}
    \centering
    \begin{tikzpicture}[node distance=2cm]
        \node[empty] (inputa) {Input A};
        \node[empty, below of=inputa] (inputb) {InputB};

        \node[empty, right of=inputa] (spacing) at ($(inputa)!0.5!(inputb)$) {};
        \node[block, right of=spacing] (register) {Register};
        \node[empty, below of=register] (writedata) {Write data};
        \node[empty, left of=writedata] (write) {Write register};
        \node[empty, right of=writedata] (writeenabled) {Write enabled};

        \node[empty, right of=inputa] (space) {};
        \node[empty, right of=space] (spacee) {};
        \node[empty, right of=spacee] (spaceee) {};
        \node[empty, right of=spaceee] (outputa) {Output A};
        \node[empty, below of=outputa] (outputb) {Output B};
        \node[empty, left of=outputb] (bspace) {};

        \path[draw, -] (inputa) -| (spacing.north);
        \path[draw, ->] (spacing.north) |- (register.155);
        \path[draw, -] (inputb) -| (spacing.south);
        \path[draw, ->] (spacing.south) |- (register.205);

        \path[draw, -] (write) |- (writedata.135);
        \path[draw, ->] (writedata.135) -- (register.225);
        \path[draw, ->] (writedata) -- (register);
        \path[draw, -] (writeenabled) |- (writedata.45);
        \path[draw, ->] (writedata.45) -- (register.315);

        \path[draw, -] (register.335) -| (bspace.north);
        \path[draw, ->] (bspace.north) |- (outputb);
        \path[draw, -] (register.25) -| (spaceee.south);
        \path[draw, ->] (spaceee.south) |- (outputa);
    \end{tikzpicture}
    \caption{The register file}
    \label{fig:register}
\end{figure}

\bf{Testing} - I start by testing if some of the initial values of the register
file is correctly set to 0. Then I test if I can write to a register, and
whether or not I can read the same value from the same address in the register.
Then, I try to write to all of the registers, except for 0, and check whether
or not the output that I get, corresponds with the output that I wrote.

\subsection{ALU}
The ALU (Arithmetic Logic Unit) is the part of the processor, which makes the
actual computation. It takes three inputs: InputA, InputB and an ALU opcode
indicating which computation to perform. It has two outputs: The result of the
computation, and a zero flag indicating whether or not the result of the
computation was 0.

I follow the approach from the book % TODO ref!
and have implemented the basic processor operations: \texttt{add}, \texttt{sub},
\texttt{and}, \texttt{or} and \texttt{slt} (set less than).

I have tested each of the four operations.
