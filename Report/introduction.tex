%Background and terminology for:
%
%CSP - hele konceptet med processer der kommunikerer
%
%SME - processer, bus og clock
%
%Machine architecture - hvertfald hvad word, bit og clock er for noget.
%
%DIKU course
%
%VHDL
%
%C\# - især hvorfor der er \texttt{uint} over det hele.
%
%beskriv hvordan figurene fungerer, i.e. box/circle = process, black wire =
%data, blue wire = control, green wire = clock
%
%Beskriv hvad MARS er, og hvordan det virker.

%install monodevelop
%clone from github
%run process T4 template on SME.Render.VHDL/Entity.tt and
%SME.Render.VHDL/TopLevel.tt
%Open example project, and press f5 to build and run

In the Machine Architecture class (ARK)~\cite{ref:ark} at the Department of
Computer Science at the University of Copenhagen (DIKU), the
theory of computer organization and design is taught. The course teaches how to
construct combinatorial circuits, which were later used to implement the
components of a MIPS (Microprocessor without Interlocked Pipeline Stages)
processor~\cite{ref:ark-book}, and how to connect the components. However, while
this course focused on the theory, it did not focus on how to construct
specialized hardware, as one might implement on an FPGA (Field Programmable
Gate Array).

In some applications, FPGAs are more attractive than general purpose CPUs, as
they do not necessarily have the same overhead in both performance and power
usage. This is due to the FPGA being more specialized towards a specific task,
where the CPU is much more complex, and as such requires additional circuitry
\cite{ref:sme-cs}.

However, the biggest problem with FPGAs is that they are programmed using
Hardware Description Languages, such as VHDL (Very high speed integrated
circuit Hardware Description Language) and Verilog. These languages are largely
unchanged from their initial design, are very tedious to work with and lack
many of the features of modern programming languages~\cite{ref:sme-cs}.

This has changed with Synchronous Message Exchange (SME)~\cite{ref:sme-cs,
ref:sme1, ref:sme2}. SME is a programming model, which is similar to the
Communicating Sequential Processes (CSP) model~\cite{ref:csp}. Both models
compute their results, by having multiple processes that communicate with each
other. The key differences are that SME is globally synchronous, has
broadcasting channels and a hidden clock. As such, SME is more suitable for
developing hardware models than CSP. Furthermore, SME can be transpiled into
VHDL~\cite{ref:sme-vhdl}, as the structure of hardware described in VHDL is
very similar to the hardware described in SME. However, SME is simpler to
implement and verify than VHDL. As such, SME gives an approachable model for
software developers, who are already familiar with the CSP model, for
generating hardware models.

\subsection*{Contribution}
In this thesis, I will combine the theory from the ARK course with the C\#
version of the SME programming model~\cite{ref:sme-cs}. As such, I will
implement a MIPS processor in SME, which can be transpiled into VHDL, and be
further synthesized onto an FPGA.

I will start by implementing some basic combinatorial circuits in SME. Then I
will combine these circuits into the basic components of the MIPS processor.
Then I will combine these components into a sincle cycle MIPS processor and
pipeline the processor, while handling data hazards and control hazards.
Finally, I go through the procedure of transpiling SME into VHDL, and how to
synhesize, place and route the transpiled VHDL onto an FPGA.

The material in this thesis could be used for teaching software developers, who
already know the CSP programming model and hardware organization, how to
construct specialized hardware.

The contents in sections~\ref{sec:logic-circuits},~\ref{sec:components},~\ref{sec:single-cycle} and~\ref{sec:pipelining} have been
published~\cite{ref:cpa-paper} to the Communicating Processes Architecture
conference~\cite{ref:cpa}.

\subsection*{Other projects}
There are several other projects, which implement a MIPS processor on an FPGA.
However, they are all written in hardware description languages VHDL and
Verilog. As such, they are closer to hardware than software and are more
difficult to understand with a background in software development. One such
project is the MIPSfpga project~\cite{ref:mipsfpga}, which is implemented in
Verilog.

There are additional projects, which gives different approaches to hardware
description languages. One of the projects is Pyrope~\cite{ref:pyrope}. Pyrope
is a modern hardware description language for live synthesis flow. I have not
used Pyrope enough to compare it to SME, but like SME, Pyrope can be transpiled into
Verilog with a testbench. As such, it should also provide a better approach for
software developers for generating hardware.
