% {{{ Preamble ----------------------------------------------------------------
\documentclass{beamer}

% encodings, fonts etc.
\usepackage[utf8x]{inputenc}
\usepackage[T1]{fontenc}

% Hold kæft utf
\makeatletter
\def\UTFviii@defined#1{%
  \ifx#1\relax
      ?%
  \else\expandafter
    #1%
  \fi
}

\makeatother

% math packages
\usepackage{amsmath, amssymb, amsthm}
\usepackage{mathtools}

% beamer configuration
\usetheme{Copenhagen}
%\useoutertheme{infolines}
%\usetheme{metropolis}
\beamertemplatenavigationsymbolsempty
%\setbeamertemplate{theorems}[numbered]
\setbeamersize{description width=2.0em}

%\usepackage{pgfpages}
%\setbeameroption{show notes}
%\setbeameroption{show notes on second screen=right}

% algorithms
\usepackage{algorithm}
\usepackage[noend]{algpseudocode}

% misc. packages
\usepackage{float}
\usepackage{varwidth}
\usepackage{listings}
\lstset{
  %breaklines=true,
  keepspaces=true,
  %frame=ltrb,
  %framesep=1pt,
  %commentstyle=\color{grey},
  basicstyle=\ttfamily\tiny,
  %numbers=left,
  title=\lstname,
  %columns=fullflexible,
  inputencoding=utf8,
  extendedchars=true,
}

% graphics and tikz
\usepackage{pgf}
\usepackage{tikz}
\usetikzlibrary{positioning,arrows,calc}
\tikzset{
    on grid,
    node distance=3cm,
    auto,
    block/.style = {
        draw,
        shape=rectangle,
        minimum height=3em,
        minimum width=3em,
        line width=1pt
    },
    control/.style = {
        draw,
        shape=circle,
        minimum height=7em,
        minimum width=3em,
        line width=1pt
    },
    mux/.style = {
        draw,
        shape=rectangle,
        minimum height=1.5em,
        minimum width=1em,
        line width=1pt
    },
    empty/.style = {
        shape=rectangle,
        minimum height=3em,
        minimum width=3em
    },
    >=latex',
}


% mathematics
\newtheorem{proposition}{Proposition}

\renewcommand{\tt}{\texttt}

% title page
\title{Pipelined MIPS processor}
\author[Carl-Johannes Johnsen]{
  \mbox{Carl-Johannes Johnsen}}
\institute{Department of Computer Science\\
           University of Copenhagen}
%\date{December 22, 2016}
% }}} -------------------------------------------------------------------------

\begin{document}

% {{{ Title page --------------------------------------------------------------
\frame{\titlepage}
% }}} -------------------------------------------------------------------------

% {{{ Table of contents -------------------------------------------------------
%\begin{frame}
%  \frametitle{Outline}
%  \tableofcontents
%\end{frame}
% }}} -------------------------------------------------------------------------
\section{Introduction}
\subsection{Overview}
\begin{frame}
    In this lecture, we will be looking at pipelining our single cycle MIPS
    processor.

    \vspace{\baselineskip}
    We will go through the motivation and background for pipelining, and the
    steps for implementing it in SME.

    \vspace{\baselineskip}
    Finally, we will look at handling the problems introduced by pipelining, by
    adding two new units: the Forwarding Unit, and the Hazard Detection Unit.
\end{frame}

\subsection{Motivation \& Background}
\begin{frame}
    The single cycle MIPS processor is not very efficient, as the clock rate is
    determined by the longest possible path in the processor.

    \vspace{\baselineskip}
    In order to increase the clock rate, we must decrease the longest path in
    the processor, by introducing pipes.
\end{frame}
\begin{frame}
    Pipes are registers in the processor, which temporarely store all the
    values computed so far.

    \vspace{\baselineskip}
    This ensures that the data does not have to travel as far, until it has
    reached a safe state.
\end{frame}
\begin{frame}
    Determining where to place the pipes, is done by dividing the processor
    into stages.

    \vspace{\baselineskip}
    We will follow the classic MIPS example, and divide the processor into 5
    stages:
    \begin{itemize}
        \item Instruction Fetch (IF)
        \item Instruction Decode (ID)
        \item Execute (EX)
        \item Memory (MEM)
        \item Write Back (WB)
    \end{itemize}
    We are going to insert a pipe between each state, i.e. 4 pipes.
\end{frame}

\section{Implementation}
\subsection{Implementation}
\begin{frame}
    We have two ways of implementing pipes in SME:
    \begin{itemize}
        \item Clocked Busses - for busses which only traverses 2 stages
        \item Clocked Processes - for dividing busses traversing more than 2
            stages
    \end{itemize}
    Just adding the \texttt{ClockedBus} attribute to the busses seems simpler.
    However, it can become more explicit, by adding additional busses, and to
    have a process, which explicitly touches all of the busses, which should go
    into the pipe.
\end{frame}
\begin{frame}
    Introducing the pipes is fairly straightforward. For each pipe, we add a
    copy of the bus, which the 'next' stage needs. Then, for each pipe, we add
    an SME process, which takes all the busses from the 'previous' stage, and
    outputs their data on the matching newly added pipe bus. Finally, the
    references in the 'next' stage should be updated to look at the piped
    busses.
\end{frame}
\begin{frame}
    This process can be repeated for all of the required pipes. There is only
    one problem: the Jump Unit. The processor do not know when to jump, until
    the MEM stage, as the adresses needs to be computed in the EX stage.

    \vspace{\baselineskip}
    To solve this, the Jump Unit should be divided out to the different stages.
    The IF stage should handle incrementing the Program Counter, and choosing
    between the addresses from the MEM stage, and the incremented Program
    Counter.

    \vspace{\baselineskip}
    The EX stage should as mentioned, compute the addresses, and finally, the
    MEM stage should hold the logic for choosing between the branch address and
    the jump address.
\end{frame}
\begin{frame}
    Finally, in the single cycle MIPS processor, we added a Write Buffer in
    order to eliminate the cycle from the Register File to the Register File.

    \vspace{\baselineskip}
    However, by introducing pipes, we have also introduced buffers, and as
    such, we can remove the Write Buffer.
\end{frame}

\subsection{Testing}



% exit section
\AtBeginSection{}
\section*{}

% {{{ Bibliography ------------------------------------------------------------
%\begin{frame}{Bibliography}
%  \tiny
%  \bibliographystyle{plain}
%  \bibliography{pl}
%\end{frame}
% }}} -------------------------------------------------------------------------

\end{document}
